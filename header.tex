\documentclass[a4paper,12pt,
		      12pt,
		      DIV=12,
		      parskip=half,
		      oneside]{scrbook}

\frenchspacing
\pagestyle{myheadings}

\usepackage[utf8]{inputenc}
\usepackage[T1]{fontenc}
\usepackage{lmodern}
\usepackage[intlimits]{amsmath} 
\usepackage{amsfonts}
\usepackage{amssymb}
\usepackage{esint}
\usepackage[dvips]{graphicx}
\usepackage{epsfig}
\usepackage{psfrag}
\usepackage[numbers,square]{natbib}
\bibliographystyle{unsrtdin}
\usepackage[pdfborder={0 0 0}]{hyperref} % Try to comment this out, if compiling crashes.
\usepackage{breakurl}
\usepackage{color} 
\definecolor{gray}{rgb}{.6,.6,.6}
\definecolor{darkgray}{rgb}{.4,.4,.4}
\definecolor{brightgray}{rgb}{.9,.9,.9}
\definecolor{darkblue}{rgb}{0,0,.8}
\definecolor{darkgreen}{rgb}{0,.7,0}
\definecolor{brightyellow}{rgb}{1,1,.5}

\usepackage{tabularx}
\usepackage{ragged2e}
\newcolumntype{Y}{>{\RaggedRight\arraybackslash}X}
\usepackage{colortbl}

\frenchspacing
\newcommand{\clearcustom}{\cleardoublepage} 

\usepackage{listings} 
\lstset{
  language=matlab,             	% the language of the code
  basicstyle=\ttfamily,		% the size of the fonts that are used for the code \footnotesize,          
  numbersep=5pt, 		% where to put the line-numbers % how far the line-numbers are from the code
  numbers=left,
  numberstyle=\tiny\color{gray},  % the style that is used for the line-numbers,
  numberbychapter=true,
  frame=lRtb,			% adds a frame around the code.  single
  rulecolor=\color{darkblue}, 	% if not set, the frame-color may be changed on line-breaks within not-black text (e.g. commens (green here))
  tabsize=2,			% sets default tabsize to 2 spaces
  showstringspaces=false,	% underline spaces within strings      
%    stringstyle=\color{red},
    commentstyle=\color{gray},
%   identifierstyle=\color{darkblue},
%   keywordstyle=\color{darkgreen},
  captionpos=b,          	% sets the caption-position to bottom
  stepnumber=1,                   % the step between two line-numbers. If it's 1, each line 
                                  % will be numbered
  backgroundcolor=\color{white},      % choose the background color. You must add \usepackage{color}
  showspaces=false,               % show spaces adding particular underscores                                  
  breaklines=true,                % sets automatic line breaking
  breakatwhitespace=true,        % sets if automatic breaks should only happen at whitespace
%   title=\lstname,                 % show the filename of files included with \lstinputlisting;
                                  % also try caption instead of title
%   escapeinside={\%*}{*)},            % if you want to add a comment within your code
%   morekeywords={*,...}               % if you want to add more keywords to the set
} 
%%%%%%%%%%%%%%%%%%%%%%%%%%%%for listing shell in linux 
\lstdefinestyle{shell}{
language=sh,
delim=[il][\bfseries]{BB}
} % for code in linux. for examp. :\begin{lstlisting}[style=shell] ....\end{listings}



\DeclareMathOperator*{\rot}{rot}
\DeclareMathOperator*{\divergence}{div}
\DeclareMathOperator*{\grad}{grad}
\newcommand{\corresponds}{\stackrel{\scriptscriptstyle\wedge}{=}}
\newcommand{\defineequal}{\stackrel{!}{=}}
\newcommand{\defineequivalent}{\stackrel{!}{\Leftrightarrow}}

\newcommand{\iconbox}[2]{%
\fbox{
\begin{minipage}{0.1\textwidth}
 \includegraphics[width=\textwidth]{#1}
\end{minipage}
\begin{minipage}{0.9\textwidth}
#2
\end{minipage}}}

\newcommand{\warning}[1]{\iconbox{svg/Warning.eps}{#1}}
\newcommand{\info}[1]{\iconbox{svg/Info.eps}{#1}}
\newcommand{\incomplete}{\warning{\textbf{Warning: This chapter or section is incomplete.}\\
Visit \url{http://openEMS.de} how to contribute and enhance this User Manual.}}

\usepackage{bm}
\usepackage{float}
\usepackage{subfig}

\newenvironment{myindentpar}% INDENT. 
     {\begin{list}{}{\setlength{\leftmargin}{1cm}}%
              \item[]% just blank 
     }
     {\end{list}}